% !TEX root = ../main.tex
% !TEX encoding = UTF-8 Unicode
% !TEX TS-program = pdflatexmk
% !TEX spellcheck = en-US
% !BIB program = biber

\chapter{Conclusion \& Future Work}\label{ch:background}

\minitoc

\section{Conclusion}

\lipsum[1]

\section{Future Work}

%% Something about the results of this thesis
This is a paragraph about the results of this thesis. This Paragraph will be written after the finalizing of the bachelor thesis.\\

This thesis applies dynamic topic modeling to social media posts in order to understand political discourse and public opinions expressed during election periods, with a particular focus on the German election.
Since social media content has an increasing amount of multi-modal data, such as images and videos, integrating these multi-modal data into the topic modeling approach presented in this thesis.
Future work can extend this approach by integrating multi-modal data, enabling a more comprehensive understanding of the evolving political discourse, as both textual and visual data can be jointly analyzed. This can reveal even deeper insights into how public opinion is formed and expressed over time.