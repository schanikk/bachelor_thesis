% !TEX root = main-msc.tex
% !TEX encoding = UTF-8 Unicode
% !TEX TS-program = pdflatexmk
% !TEX spellcheck = en-US
% !BIB program = biber

%%%%%%%%%%%%%%%%%%%%%%%%%%%%%%%%%%%%%%%%%%%%%%%%%%%%%%%%%%%%%%%
%% UHH LT THESIS TEMPLATE based on the OXFORD THESIS TEMPLATE
%% see main.tex for explanations and options

%%%%% CHOOSE PAGE LAYOUT
% PDF output (ie equal margins, no extra blank pages, for online publication):
\documentclass[hidelinks, a4paper, nobind, msc]{conf/uhhltthesis}

% Your full degree name.
\degree{Master of Science (M.\ Sc.)}

% declaration of academic honesty for your specific major,
% NOTE: this has to be adjusted to the requirements of your students office!!
\affidavit{% declaration of academic honesty
\par\noindent Hiermit versichere ich an Eides statt, dass ich die vorliegende Arbeit im Bachelorstudiengang Informatik selbstständig verfasst und keine anderen als die angegebenen Hilfsmittel – insbesondere keine im Quellenverzeichnis nicht benannten Internet-Quellen – benutzt habe.
Alle Stellen, die wörtlich oder sinngemäß aus Veröffentlichungen entnommen wurden, sind als solche kenntlich gemacht.
Ich versichere weiterhin, dass ich die Arbeit vorher nicht in einem anderen Prüfungsverfahren eingereicht habe.
%
\vspace*{2em}%
%
\par\noindent I hereby declare in lieu of an oath that I have written this thesis for the Bachelor's degree programme in Computer Science independently and have not used any aids other than those specified - in particular no Internet sources not named in the list of sources.
All passages taken verbatim or in spirit from publications are labelled as such.
I further certify that I have not previously submitted the thesis in another examination procedure.

% Translated with DeepL.com (free version)}

% outsource configurations
% !TEX root = ../main.tex
% !TEX encoding = UTF-8 Unicode
% !TEX TS-program = pdflatexmk
% !TEX spellcheck = en-US
% !BIB program = biber

\usepackage[ngerman,main=english]{babel}
\input{conf/font-conf}

%%%%% SELECT YOUR DRAFT OPTIONS
%\degreedate{April, 2024}
%\degreedate{~}
\degreedate{\emph{DRAFT Printed on \today}}

% Footer
\fancyfoot[C]{\emph{DRAFT Printed on \today}}

% use corrections as delta highlighting mechanism (see ociamthesis)
% This highlights (in blue) corrections marked with (for words) \mccorrect{blah} or (for whole
% paragraphs) \begin{mccorrection} . . . \end{mccorrection}.
%\correctionstrue

% BIBLIOGRAPHY resources
\addbibresource{./bib/biblio-clean.bib}
%\addbibresource{./bib/biblio-temp.bib}

% Uncomment this if you want equation numbers per section (2.3.12), instead of per chapter (2.18):
%\numberwithin{equation}{subsection}

%%%%% THESIS / TITLE PAGE INFORMATION
\newcommand{\mytitle}{Fancy \\ Thesis \\ Title}
\title{\mytitle}

% title without line breaks
\newcommand{\mytitleplain}{Fancy Thesis Title}
\titleplain{\mytitleplain}

% authorname
\newcommand{\authorname}{Allan M. Turing}

% specifiy Date of submission
\dateofsubmission{ 1.1.2099 }

% specifiy Date of disputation if necessary (usually not necessary for bachelor or master degree)
\dateofdisputation{ 2.1.2099 } 

% who's your supervisor?
\supervisonby{John von Neumann, Universität Hamburg}

% who's part of your comittee?
\comittee{%
  1\textsuperscript{st} Examiner: Prof.\ Dr.\ Chris Biemann, Universität Hamburg \\%
  2\textsuperscript{nd} Examiner: Dr.\ Konrad Zuse, Universität Hamburg \\%
}%
%% OVERRIDE defaults:
%\university{Universit\"a{}t Hamburg}%
%\unviersitylogo{\includegraphics[width=\textwidth,clip,trim=1.95cm 2.5cm 1.95cm 2.5cm]{figures/up-uhh-logo-u-2010-u-farbe-u-cmyk-modus}}
%\address{Hamburg, Germany}%
\faculty{Faculty of Mathematics, Informatics and Natural Sciences}
%\facultylogo{\includegraphics[width=\textwidth]{figures/uhh-min-faculty-de}}
% \department{Department of Informatics}
% \researchgroup{Language Technology}
%\grouplogo{\includegraphics[width=\textwidth]{figures/uhh-lt-logo}}


% set author
\author{\authorname}

%%%%% PDF METADATA
\hypersetup{%
  pdftitle={\mytitleplain},%
  pdfauthor={\authorname},%
  pdfsubject={\mytitleplain},%
  pdfview=FitH,%
  pdfstartview=FitV,%
  pdfproducer={\authorname},%
  pdfcreator=\textsc{LaTeX},%
	colorlinks=true,% remove boxes for links
	citecolor=black,%
	linkcolor=LimeGreen,% black % NOTE: make all colors black for the final print submission
	anchorcolor=LimeGreen,% black %
	filecolor=LimeGreen,% black %
	runcolor=LimeGreen,% black %
	urlcolor=LimeGreen,% black %
	menucolor=LimeGreen,% black %
}%

%%%%% MORE MACROS
\usepackage{makeidx}
\usepackage{placeins}
\usepackage[labelfont={footnotesize,up,tt}]{subcaption} % subrefformat=parens,
\usepackage{kantlipsum}
\usepackage{lipsum}
\usepackage{url}
\usepackage{amssymb}
\usepackage{amsmath}
\usepackage{booktabs}
\usepackage{listings}
\usepackage{amsthm}
\theoremstyle{definition}
\newtheorem{definition}{Definition}%[section]

%%%%% MORE COMMANDS, REDEFINES, ENVIRONMENTS, ...

% define hangindent for listitems
\newlength\listindent
\setlength\listindent{13pt}
\newcommand{\hangindentlistitems}{\parshape 2 0cm \linewidth \listindent \dimexpr\linewidth-\listindent\relax}

%%%
\graphicspath{%
  {figures/}%
  {figures/ch-1/}%
  {figures/sample/}%
}%

%% manual hyphenation rules
\hyphenation{down-stream down-stream-tasks}


\makeindex

%%%%% THE ACTUAL DOCUMENT STARTS HERE
\begin{document}

%%%%% CHOOSE YOUR LINE SPACING HERE
% Zeilenabstand Einstellung
\setlength{\textbaselineskip}{\baselineskip}
\setlength{\frontmatterbaselineskip}{\baselineskip}

% Pages are roman numbered from here, though page numbers are invisible until ToC
\begin{romanpages}

% Title page is created here (includes affidavit)
\maketitle

%%%%% DEDICATION -- If you'd like, un-comment the following.
\begin{dedication}
  This work is dedicated to some important person(s) for some important reason
\end{dedication}

%%%%% ACKNOWLEDGEMENTS -- If you'd like, un-comment the following.
\begin{acknowledgements}
  \input{text/acknowledgements}
\end{acknowledgements}

%%%%% THEMED QUOTE -- If you'd like one, un-comment the following.
\begin{themedquote}{Johann Wolfgang von Goethe, 1829}
  Alles Gescheite ist schon gedacht worden.\\
Man muss nur versuchen, es noch einmal zu denken. \\[\baselineskip]

All intelligent thoughts have already been thought;\\
what is necessary is only to try to think them again.
\end{themedquote}

%%%%% ABSTRACT -- Nothing to do here except comment out if you don't want it.
\begin{abstract}
  \input{text/abstract}
\end{abstract}

%%%%% ABSTRACT IN GERMAN
\begin{germanabstract}
  \input{text/zusammenfassung}
\end{germanabstract}

% This aligns the bottom of the text of each page.  It generally makes things look better.
\flushbottom
% This is where the whole-document ToC appears:
{%
  \setcounter{page}{0}%
  %
  \hypersetup{%
	  linkcolor=black%
  }%
  \tableofcontents%
}%
% This aligns the bottom of the text of each page.  It generally makes things look better.
\flushbottom


% end roman page numbering
\end{romanpages}

% This aligns the bottom of the text of each page.  It generally makes things look better.
\flushbottom

%%%%% CHAPTERS
% !TEX root = ../main.tex
% !TEX encoding = UTF-8 Unicode
% !TEX TS-program = pdflatexmk
% !TEX spellcheck = en-US
% !BIB program = biber


% want quotes?
%\begin{savequote}[8cm]
%Computers are incredibly fast, accurate and stupid; humans are incredibly slow, %inaccurate, and brilliant; together they are powerful beyond imagination
%
%\qauthor{--- Albert Einstein}
%
%\end{savequote}

\chapter{Introduction}\label{ch:1-intro}%
%

% here goes the content
In this chapter of the thesis, we will first cover the motivation of our thesis and the importance of natural language processing, with a specific focus on topic modeling.
Following the motivation, we will discuss the problem that this thesis aims to address and solve, as well as the research questions we seek to answer.

\section{Motiviation}
While social media platforms like Facebook, Twitter, and Instagram were initially employed for personal interactions among family and friends. In the past decade, they have evolved into influential spaces for political discourse and public opinion placement.
Especially during elections, social media platforms, in recent years, have been excessively used as a platform for political discourses by users.

In Germany, recent events such as the the dissolution of the governing coalition and the recent events in the German economy as well as events in Europe have further polarized public opinion.
Especially on social media platforms, this division is even more visible than ever before, and politicians leverage social media platforms to generate content and distribute their viewpoints on different topics to run election campaigns and attract voters for their party.
The educational initiative AI4Democracy highlights the transformative role of AI in policymaking. Analyzing social media data during elections aims to bridge the gap between policymakers and citizens in order to strengthen the democratic process. Also, by applying AI to analyzing social media content, voters concern can be made more accessible for policymakers, enabling them to understand the public discource better.
This thesis seeks to leverage AI techniques to address the increasing volume and complexity of political discourse during elections.

During election periods, social media platforms are excessively used by users for political discourse and forming opinions. This vast amount of data cannot be used by an individual alone for policy-making and analyzing voters' concerns.
However, in recent years, Natural Language Processing (NLP) advanced significantly due to the theoretical advancement through the transformer architecture, as well as the technical advances such as the incredible improvement of the utilization of Graphics processing units (GPU) in deep learning architectures.
Natural Language Processing (NLP) is now being adopted in almost every field, and enables a horizontal layer across all domains.
These developments made it possible to handle large textual datasets and extract richer semantic information from text.
By applying Natural Language Processing (NLP), this amount of data can be used, to gain insights into the voter's interests and concerns, and understand how the public responds during the election period in Germany. Topic modeling is a tool that can be used to analyze different topics, patterns, and hidden structures within large collections of textual data. This makes it useful for studying how the topic frequency changes over time and capturing changes in the content of different topics within the election period.

\section{Problem Statement}\label{ch:1-problem}

Existing research and studies primarily focus on static analytics during elections. Thus they do not incorporate how the political discourse changes over time during election campaigns. Static Analysis makes it difficult to capture temporal changes of public opinions, especially with the current development of the political discourse, which has shown increased usage of populist debates.
This thesis tries to solve this gap by applying time-sensitive topic modeling to election-related social media data. The problem is to analyze how political topics evolve over time during the election campaign. Understanding these patterns can help policymakers understand the citizen's concerns, and understand the impact of external events on the public narrative.


\subsection{Research Questions}

The problem, mentioned in Section \ref{ch:1-problem}, opens up several research questions. Since most analyses on social media platforms perform static topic modeling on generated content, this paper will try to answer the question of how topic modeling can be used to analyze temporal content during election periods, especially during the German election period until the early parliamentary elections on the 23rd of February 2025. This opens up the first research questions, which can be formulated as follows: \\

\textbf{1. How can topic modeling be utilized on temporal textual data to analyze public opinions during election campaigns?}\\

This Research question opens up further questions, such as:\\

\textbf{2. How does the importance of political topics change over time during the official election?}\\


Answering these questions in this thesis should help policymakers and journalists better understand the concerns and opinions of the public and improve the democratical understanding.

%3. How do external events influence the political discourse on social media platforms?\\


\subsection{Structure of this Thesis}

In the following chapters of this thesis, we will first familiarize us with the required theoretical knowledge needed, to 

%\begin{table}
%  \centering
%  \begin{tabular}{lll}
%    \toprule
%     & A & B \\
%    \midrule
%    C & 1 & 2 \\
%    D & 3 & 4 \\
%    \bottomrule
%  \end{tabular}\caption[shorter caption]{potentially very long %caption}\label{tab:sample}
%\end{table}

%%%% %%%%%%%%% %%%%%%%%% %%%%%%%%% %%%%%%%%%
%%%%%%%%% %%%%%%%%% %%%%%%%%% %%%%%%%%% %%%%

\include{text/ch-2}

%% APPENDICES %%
% Starts lettered appendices, adds a heading in table of contents, and adds a
%    page that just says "Appendices" to signal the end of your main text.
\startappendices
% Add or remove any appendices you'd like here:
\include{text/appendix-1}

%%%%% REFERENCES
% use single-space References
\setlength{\baselineskip}{0pt}
{%
	\renewcommand*\MakeUppercase[1]{#1}%
	% \nocite{*}
	\printbibliography[heading=bibintoc,title={\bibtitle}]
}%

\end{document}
